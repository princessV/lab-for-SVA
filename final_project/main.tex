\documentclass[journal]{IEEEtran}


\hyphenation{op-tical net-works semi-conduc-tor}

\usepackage{graphicx}
\graphicspath{{public/}}

\begin{document}

\title{Vulnerability Prediction Modeling Based on Machine Learning and Natural Language Processing}
\author{Qiyang~Gu, Ujjawal~Sharma, Zhiyuan~Li}
\maketitle


\begin{abstract}
To assist the vulnerability identification process, many researchers proposed prediction models that highlight(for inspection) the most likely to be vulnerable parts of a software system. The normal method for vulnerability prediction modeling(VPM), users need to manually analyze the source code of the software, which requires very tedious work. In general, the manual analyzation methods requires analyzing code size, structural complexity, information flow complexity, and so on, which means users must be able to get access to the source code of the software system.

In this project, we introduce a new method to conduct the vulnerability prediction modeling, that is a deep learning based framework that assesses software vulnerabilities from the natural language processing. This approach starts from the name and the version of the software, using network crawler to crawl security information in the security database on the Internet, then utilizes NLP technology on big data on the Internet to predict software security indicators like the amount, impact and category of the security vulnerabilities in the software, therefore making evaluations on software security. This method based on the security information in NVD(National Vulnerability Database), using Bing to search software security information, utilizing machine-learning technology to generate security evaluation report.
\end{abstract}

\begin{IEEEkeywords}
 Vulnerability Prediction Modeling, Machine Leaning, Natural Language Processing
\end{IEEEkeywords}


\section{Introduction}
\IEEEPARstart{F}or software system requires that high security and high quality, a vulnerability prediction model is required to give the software makers the understanding of software quality and reduce the risk of being attacked. A good vulnerability prediction model can prevent mistakes and risks from occurring and avoid a ton of losses of software makers.

Existing software security assessment methods are usually aimed at the software's source code. With the increase of software size and complexity, such methods gradually become difficult to meet the requirements of safety assessment. For many closed-source software systems, because assessors cannot get the source code, it is difficult to use this type of method for safety assessment analysis. In fact, many online security databases and vulnerability forums contain a large amount of raw information related to software security. At present, this information is not fully used, therefore, this information and the corresponding experience of using the software as the basis for evaluation will provide a new idea for software vulnerability prediction modeling.

In this project, we introduce a software VPM method based on machine learning and natural language processing. The method starts from the software name and version number provided by the user, crawling the online safety database information, and combining the natural language analysis of web big data. Through the prediction of the number of software vulnerabilities, the severity, and the main types, to achieve this method for software VPM.


\section{Reasons and Goals}
The reason for this project or lecture is to introduce a new method for software vulnerability prediction modeling, which based on machine learning and natural language processing. We hold this opinion that a good and effective software vulnerability prediction model is very helpful and meaningful for the software makers to improve the quality of software and reduce the risk of being attacked so that it can improve the profitability of companies. So it is important to find an effective method for vulnerability prediction model, it not only can be a benefit for the software makers but also can be more secure for the entire software environment and a more secure environment for Internet users. Another incentive for us to introduce this method is that many online security databases and vulnerability forums contain a large amount of raw information related to software security. At present, this information is not fully used, therefore, this information and the corresponding experience of using the software as the basis for evaluation will provide a new idea for software vulnerability prediction modeling. 

Below are the goals, which we had in our mind while introducing this VPM method:
\begin{itemize}
\item To understand what is vulnerability prediction model.
\item To understand how the mechine learning method can be applied in this VPM.
\item To understand the way that natural language processing method applied in this VPM.
\item To get a good conceptual grip on what each step of collecting data from NVD.
\item To know how to use the mechine learning algorithm(random forest) to train the data.
\end{itemize}

\section{Project Execution Highlights}
The main process of our VPM is shown below. \clearpage
\begin{figure}[!htbp]
\centering
\includegraphics[height=4.5cm]{VPM_1}
\caption{Main process}
\end{figure}
As shown in the figure, the core of our method is a machine learning model. The method includes an offline model training process and an online security assessment process, corresponding to the upper and lower half of the graph. In the model training process, we first obtain the software security information and software security merits from the training data source NVD. Then, we combine the predefined software feature words with the software name and version number, searched in the search engine, and we use crawlers to crawl the contents of the searched webpage to obtain the software security information on the Internet. We take the software security information on the network as the input of the model and the software security merits as the output of the model to train the model. In the security assessment process, we enter the software name and version number combined with the feature words in the search engine, and use the crawler to crawl the contents of the search results page to get the software security information. Finally, the software security information is used as the input of the model. The model outputs the security merits of the software and generates the security assessment report of the software.

We will discuss it detailedly in the following.
\subsection{Pre-process the trainning data}
Pre-processing the training data is the process of preprocessing the data in NVD and taking the software security information and expected output from the NVD. Software security information includes information about vulnerabilities in the NVD and information in related external links. We use the xml parser to analyze the dataset in the xml document provided by the NVD to obtain the vulnerability information in the NVD. Then we use the xml parser to get the address of the vulnerability-related external link from the xml document, selecting the URLs of some of these links for easy-to-understand, well-formed secure database sites, crawling for vulnerabilities on those sites. Expected output is the software's security merits, including the number, severity, and major types of software vulnerabilities. We use xml parser to analyze the dataset of xml document provided by NVD and take out the severity and the type of each vulnerability. After all the vulnerability information is stored in the database, we use sql statement to calculate the number, severity and the main type of vulnerability in each software.


\subsection{Corpus collection}
Corpus collection is the process of using search engines to collect software security information on the Internet. Search keywords is the name and version of the software combined with the pre-defined feature words. The reason of the usage of the feature words it to filter irrelevant sites. Firstly, we define the high-frequency words appears in the vulnerability in the NVD as the initial candidate feature words, and then use different combinations of these candidate feature words as the candidate feature phrases to train and evaluate the models respectively, and select the candidate features with the highest accuracy Word combination as the final feature word.

\subsection{Model preparation and training}
Model preparation and training is the process of preparing the training data of the machine learning model and training the model. We use the software security descriptive information in the training data source to train the neural network, and use the trained neural network to convert the software security information on the Internet into the vector as the input of the model, the software security merits as the expected output to train the model.

\section{Findings}
In order to evaluate the accuracy of our VPM, we design three experiments to compare the prediction based on our models with the actual value stored in the NVD.

We define some meatures for our experiment, including $CRE$ (Corrected Relative Error), $FRank$. 


\section{Conclusion}
In this project and lecture, we introduce a software VPM method based on machine learning and natural language processing. The method starts from the software name and version number provided by the user, crawling the online safety database information, and combining the natural language analysis of web big data. Through the prediction of the number of software vulnerabilities, the severity, and the main types, to achieve this method for software VPM. The method is based on the security information of the National Vulnerability Database of the United States and uses Bing search engine to search the software related vulnerability data. Based on the machine learning technology, the method generates the safety assessment report of the corresponding software, which is a new software vulnerability prediction model method. 

We implemented the method based on Scala and the Java language and used the vulnerability data from the NVD database to evaluate the predictive performance of the method. The evaluation results show that the proposed method can make an accurate vulnerability prediction model for most software. And we compared the evaluation accuracy of different training algorithms, search engines, and feature word methods. The results show that the random forest algorithm, Bing search engine, and the combination of three specific feature words have good evaluation results.

\section{Appendix}
We put all supplementary information to the Wiki for our Gitlab project.

\begin{thebibliography}{10}
\bibitem{defeating_tech}
N. Stack Overflows - Defeating Canaries, ``Stack Overflows - Defeating Canaries, ASLR, DEP, NX'', Security.stackexchange.com. [Online]. Available: https://security.stackexchange.com/questions/20497/stack-overflows-defeating-canaries-aslr-dep-nx.

\bibitem{dep}
``Executable space protection'', En.wikipedia.org. [Online]. Available: https://en.wikipedia.org/wiki/Executable\_space\_protection.
\end{thebibliography}


\end{document}