\documentclass[a4paper, 10pt]{article}

\usepackage{graphicx}
\graphicspath{{public/}}

\usepackage{geometry}
\geometry{left=1.5cm, right=1.5cm, top=3cm, bottom=2cm}

\usepackage{array}
\usepackage{listings}
\usepackage{color}


\newcommand{\tabincell}[2]{\begin{tabular}{@{}#1@{}}#2\end{tabular}}

\lstset{
    backgroundcolor=\color[RGB]{245,245,244}    
}


\begin{document}
\title{\textbf{Appendix}}
\author{\textit{Qiyang Gu, Ujjawal Sharma, Zhiyuan Li}}
\date{}
\maketitle



\section{From Linux Command Line}
\subsection{Compile the binary}

We started the Reverse Engineering process by downloading the WebBench source code from the below URL:
\begin{lstlisting}
https://github.com/wxmyt/WebBench   
\end{lstlisting}

\noindent
We downloaded the file to a folder and unzipped the file 

\begin{figure}[!htbp]
\centering
\includegraphics[height=10cm]{Unzip.png}
\caption{Unzipping the downloaded code}
\end{figure}

\noindent
After unzipping the source code, we observed that the source code is in C programming language. For the reverse engineering process we used the Ubuntu 14.04 Operating System. We also observed that the source code was using the `Make' build automation tool for building the source code. `Make' is a build automation tool that automatically builds executable programs and libraries from the source code by reading files called `Makefile' which is provided with the source code and contains the information on how to derive the target application from sourcecode. \\

\noindent
As the source code was in C programming language, it was important to use the GCC compiler as well. For installing both `Make' and `GCC' compiler we ran the following code on terminal.

\begin{lstlisting}
 $ sudo apt-get install build-essential  
\end{lstlisting}

\noindent
After the installation of `Make' and `GCC' was done, we navigated to the unzipped WebBench folder and ran the below command to build the WebBench binary:
\newpage
\begin{figure}[!htbp]
\centering
\includegraphics[height=10cm]{BeforeMake.png}
\caption{Image showing the source code folder before running make command}
\end{figure}

\noindent
The make command ran successfully with 1 warning which we ignored because that came with the source code. We can also observe in the screenshot below that the `Make' command created the final executable for the application `WebBench':

\begin{figure}[!htbp]
\centering
\includegraphics[height=10cm]{AfterMake.png}
\caption{Image showing the source code folder after running make command}
\end{figure}
\noindent

\subsection{Dump the headers}
A header file is a file containing C declarations and macro definitions to be shared between several source files.We can request the use of any header file in our C programme by using the `include' keyword. Header files declare the interfaces to parts of the operating system. We can include them in our programme to supply the definitions and declarations we need to invoke system calls and libraries. During the development process a developer creates a header file which contains the declarations of interfaces between the source files of the program. Whenever we have a group of related declarations and macro definitions, all or most of which are needed in several different source files, it is a good idea to create a header file for them. \\

\noindent
For dumping the headers, we use the objdump programme provided by the GCC compiler. It is used for displaying various information about object files. It is used in the process of reverse engineering to view an executable in assembly form as a disassembler. objdump converts the binary file to a more human readable format by converting it to the Hex-form. Using the result obtained on the screen we can analyse how the binary is interacting with the CPU.\\

\noindent
For getting the dump of the headers, we executed the following command to dump the headers:

\begin{lstlisting}
$ objdump -h webbench
\end{lstlisting}

\noindent
We get the following headers.
{
\normalsize
\begin{lstlisting}
Idx Name          Size      VMA               LMA               File off  Algn
  0 .interp       0000001c  0000000000400238  0000000000400238  00000238  2**0
                  CONTENTS, ALLOC, LOAD, READONLY, DATA
  1 .note.ABI-tag 00000020  0000000000400254  0000000000400254  00000254  2**2
                  CONTENTS, ALLOC, LOAD, READONLY, DATA
  2 .note.gnu.build-id 00000024  0000000000400274  0000000000400274  00000274  2**2
                  CONTENTS, ALLOC, LOAD, READONLY, DATA
  3 .gnu.hash     00000030  0000000000400298  0000000000400298  00000298  2**3
                  CONTENTS, ALLOC, LOAD, READONLY, DATA
  4 .dynsym       000003f0  00000000004002c8  00000000004002c8  000002c8  2**3
                  CONTENTS, ALLOC, LOAD, READONLY, DATA
  5 .dynstr       0000019e  00000000004006b8  00000000004006b8  000006b8  2**0
                  CONTENTS, ALLOC, LOAD, READONLY, DATA
  6 .gnu.version  00000054  0000000000400856  0000000000400856  00000856  2**1
                  CONTENTS, ALLOC, LOAD, READONLY, DATA
  7 .gnu.version_r 00000050  00000000004008b0  00000000004008b0  000008b0  2**3
                  CONTENTS, ALLOC, LOAD, READONLY, DATA
  8 .rela.dyn     00000060  0000000000400900  0000000000400900  00000900  2**3
                  CONTENTS, ALLOC, LOAD, READONLY, DATA
  9 .rela.plt     00000378  0000000000400960  0000000000400960  00000960  2**3
                  CONTENTS, ALLOC, LOAD, READONLY, DATA
 10 .init         0000001a  0000000000400cd8  0000000000400cd8  00000cd8  2**2
                  CONTENTS, ALLOC, LOAD, READONLY, CODE
 11 .plt          00000260  0000000000400d00  0000000000400d00  00000d00  2**4
                  CONTENTS, ALLOC, LOAD, READONLY, CODE
 12 .plt.got      00000008  0000000000400f60  0000000000400f60  00000f60  2**3
                  CONTENTS, ALLOC, LOAD, READONLY, CODE
 13 .text         00000fe2  0000000000400f70  0000000000400f70  00000f70  2**4
                  CONTENTS, ALLOC, LOAD, READONLY, CODE
 14 .fini         00000009  0000000000401f54  0000000000401f54  00001f54  2**2
                  CONTENTS, ALLOC, LOAD, READONLY, CODE
 15 .rodata       00000b84  0000000000401f60  0000000000401f60  00001f60  2**5
                  CONTENTS, ALLOC, LOAD, READONLY, DATA
 16 .eh_frame_hdr 00000054  0000000000402ae4  0000000000402ae4  00002ae4  2**2
                  CONTENTS, ALLOC, LOAD, READONLY, DATA
 17 .eh_frame     000001bc  0000000000402b38  0000000000402b38  00002b38  2**3
                  CONTENTS, ALLOC, LOAD, READONLY, DATA
 18 .init_array   00000008  0000000000602e10  0000000000602e10  00002e10  2**3
                  CONTENTS, ALLOC, LOAD, DATA
 19 .fini_array   00000008  0000000000602e18  0000000000602e18  00002e18  2**3
                  CONTENTS, ALLOC, LOAD, DATA
 20 .jcr          00000008  0000000000602e20  0000000000602e20  00002e20  2**3
                  CONTENTS, ALLOC, LOAD, DATA
 21 .dynamic      000001d0  0000000000602e28  0000000000602e28  00002e28  2**3
                  CONTENTS, ALLOC, LOAD, DATA
 22 .got          00000008  0000000000602ff8  0000000000602ff8  00002ff8  2**3
                  CONTENTS, ALLOC, LOAD, DATA
 23 .got.plt      00000140  0000000000603000  0000000000603000  00003000  2**3
                  CONTENTS, ALLOC, LOAD, DATA
 24 .data         00000020  0000000000603140  0000000000603140  00003140  2**3
                  CONTENTS, ALLOC, LOAD, DATA
 25 .bss          000008c0  0000000000603160  0000000000603160  00003160  2**5
                  ALLOC
 26 .comment      00000034  0000000000000000  0000000000000000  00003160  2**0
                  CONTENTS, READONLY
 27 .debug_aranges 00000030  0000000000000000  0000000000000000  00003194  2**0
                  CONTENTS, READONLY, DEBUGGING
 28 .debug_info   000029a7  0000000000000000  0000000000000000  000031c4  2**0
                  CONTENTS, READONLY, DEBUGGING
 29 .debug_abbrev 00000481  0000000000000000  0000000000000000  00005b6b  2**0
                  CONTENTS, READONLY, DEBUGGING
 30 .debug_line   0000057b  0000000000000000  0000000000000000  00005fec  2**0
                  CONTENTS, READONLY, DEBUGGING
 31 .debug_str    000008d0  0000000000000000  0000000000000000  00006567  2**0
                  CONTENTS, READONLY, DEBUGGING
 32 .debug_loc    0000194e  0000000000000000  0000000000000000  00006e37  2**0
                  CONTENTS, READONLY, DEBUGGING
 33 .debug_ranges 000000c0  0000000000000000  0000000000000000  00008785  2**0
                  CONTENTS, READONLY, DEBUGGING    
\end{lstlisting}
}

\noindent
In the output above, Size is the size of the loaded section, VMA represents the virtual memory address, LMA represents the logical memory address, File off is this section’s offset from the beginning of the file, Algn represents alignment, CONTENTS, ALLOC, LOAD, READONLY, DATA are flags that represent that a particular section is to be LOADED or is READONLY etc. \cite{objdump}\\

\noindent
As the screen shot showing above, there are total of 33 section headers(only partial output is shown). These sections are used by the system to hold program and control information. \\

\noindent
For example, the function of the section .iterp is that it holds the path name of a program interpreter. If the file has a loadable segment that includes the section, the section’s attributes will include the SHF\_ALLOC bit; otherwise, that bit will be off.


\subsection{Analyzing the library dependencies}
A Shared Library is a file that is intended to be shared by executable files. Modules used by a program are loaded from individual shared objects into memory at load time or run time, rather than being copied by a linker when it creates a single monolithic executable file for the program. All the code relating to the library is in this file and it is directly linked into the program at the run time. \\

\noindent
We used the following command to print the shared library dependencies which are loaded by any running processes:
\begin{lstlisting}
$ ldd webbench
\end{lstlisting}
By running the above command, we got the below library dependencies.

\begin{lstlisting}
linux-vdso.so.1 =>  (0x00007ffd99184000)
libc.so.6 => /lib/x86_64-linux-gnu/libc.so.6 (0x00007fd9627f2000)
/lib64/ld-linux-x86-64.so.2 (0x0000562626363000)
\end{lstlisting}

\noindent
linux-vdso.so.1 is the Linux Virtual Dynamic Shared Object. This file exists in each executables address space. \\
libc.so.6 is libc shared object version 6, which sits in /lib/i386-linux-gnu and its memory location is 0x00007ffd99184000. \\
The last line talks about ld-linux.so version 2 under /lib64. This will find and load shared libraries node needs. It will prepare those libraries and run them.



\subsection{List the symbols}
We run the following command to list the symbols.
\begin{lstlisting}
$ nm webbench
\end{lstlisting}
We get these symbols.

{
\normalsize
\begin{lstlisting}
                 U alarm@@GLIBC_2.2.5
0000000000401066 t alarm_handler
0000000000401162 t benchcore
0000000000603150 D benchtime
0000000000603160 B __bss_start
00000000006031a4 B bytes
0000000000603158 D clients
                 U close@@GLIBC_2.2.5
0000000000603188 b completed.7585
                 U connect@@GLIBC_2.2.5
0000000000603140 D __data_start
0000000000603140 W data_start
0000000000400fa0 t deregister_tm_clones
0000000000401020 t __do_global_dtors_aux
0000000000602e18 t __do_global_dtors_aux_fini_array_entry
0000000000603148 D __dso_handle
0000000000602e28 d _DYNAMIC
0000000000603160 D _edata
0000000000603a20 B _end
                 U exit@@GLIBC_2.2.5
00000000006031a8 B failed
                 U fclose@@GLIBC_2.2.5
                 U fdopen@@GLIBC_2.2.5
0000000000401f54 T _fini
000000000060319c B force
0000000000603198 B force_reload
                 U fork@@GLIBC_2.2.5
                 U __fprintf_chk@@GLIBC_2.3.4
0000000000401040 t frame_dummy
0000000000602e10 t __frame_dummy_init_array_entry
0000000000402cf0 r __FRAME_END__
                 U fwrite@@GLIBC_2.2.5
                 U gethostbyname@@GLIBC_2.2.5
                 U getopt_long@@GLIBC_2.2.5
0000000000603000 d _GLOBAL_OFFSET_TABLE_
                 w __gmon_start__
0000000000402ae4 r __GNU_EH_FRAME_HDR
00000000006031e0 B host
000000000060315c D http10
                 U index@@GLIBC_2.2.5
                 U inet_addr@@GLIBC_2.2.5
0000000000400cd8 T _init
0000000000602e18 t __init_array_end
0000000000602e10 t __init_array_start
0000000000401f60 R _IO_stdin_used
                 U __isoc99_fscanf@@GLIBC_2.7
                 w _ITM_deregisterTMCloneTable
                 w _ITM_registerTMCloneTable
0000000000602e20 d __JCR_END__
0000000000602e20 d __JCR_LIST__
                 w _Jv_RegisterClasses
0000000000401f50 T __libc_csu_fini
0000000000401ee0 T __libc_csu_init
                 U __libc_start_main@@GLIBC_2.2.5
0000000000402900 r long_options
0000000000401310 T main
                 U __memcpy_chk@@GLIBC_2.3.4
00000000006031a0 B method
00000000006031c0 B mypipe
0000000000603168 B optarg@@GLIBC_2.2.5
0000000000603160 B optind@@GLIBC_2.2.5
                 U perror@@GLIBC_2.2.5
                 U pipe@@GLIBC_2.2.5
                 U __printf_chk@@GLIBC_2.3.4
0000000000603190 B proxyhost
0000000000603154 D proxyport
                 U putchar@@GLIBC_2.2.5
                 U puts@@GLIBC_2.2.5
                 U read@@GLIBC_2.2.5
0000000000400fe0 t register_tm_clones
0000000000603220 B request
                 U setvbuf@@GLIBC_2.2.5
                 U shutdown@@GLIBC_2.2.5
                 U sigaction@@GLIBC_2.2.5
                 U sleep@@GLIBC_2.2.5
0000000000401095 T Socket
                 U socket@@GLIBC_2.2.5
00000000006031ac B speed
                 U __stack_chk_fail@@GLIBC_2.4
0000000000400f70 T _start
0000000000603180 B stderr@@GLIBC_2.2.5
                 U __strcat_chk@@GLIBC_2.3.4
                 U strcat@@GLIBC_2.2.5
                 U strchr@@GLIBC_2.2.5
                 U strncasecmp@@GLIBC_2.2.5
                 U __strncpy_chk@@GLIBC_2.3.4
                 U strrchr@@GLIBC_2.2.5
                 U strstr@@GLIBC_2.2.5
                 U strtol@@GLIBC_2.2.5
00000000006031b0 B timerexpired
0000000000603160 D __TMC_END__
0000000000401071 t usage
                 U write@@GLIBC_2.2.5
\end{lstlisting}
}
\noindent
We can look at the symbol table of the binary file of Webbench and find that it shows about ninety symbols, which means that the Webbench maybe has included about ninety functions(including calling libc) in its source code. When the Webbench program starts running, it goes to \_start at first, then goes to main, and then goes to other functions of the program.


\subsection{Dump the vm tables}
We fistly run the webbench, then we use \textit{ps} command to see the pid, and then we dump the vm tables. 
\begin{lstlisting}
$ ./webbench -f http://abc.go.com/ -t 100
$ ps -a | grep webbench
$ cat /proc/4025/maps
\end{lstlisting}
We get these following vm tables.
{
\scriptsize
\begin{lstlisting}
00400000-00403000 r-xp 00000000 08:01 414416  /home/qiyanggu/Documents/WebBench/webbench
00602000-00603000 r--p 00002000 08:01 414416  /home/qiyanggu/Documents/WebBench/webbench
00603000-00604000 rw-p 00003000 08:01 414416  /home/qiyanggu/Documents/WebBench/webbench
00a5a000-00a7b000 rw-p 00000000 00:00 0       [heap]
7f9b2a179000-7f9b2a190000 r-xp 00000000 08:01 399119  /lib/x86_64-linux-gnu/libresolv-2.23.so
7f9b2a190000-7f9b2a390000 ---p 00017000 08:01 399119  /lib/x86_64-linux-gnu/libresolv-2.23.so
7f9b2a390000-7f9b2a391000 r--p 00017000 08:01 399119  /lib/x86_64-linux-gnu/libresolv-2.23.so
7f9b2a391000-7f9b2a392000 rw-p 00018000 08:01 399119  /lib/x86_64-linux-gnu/libresolv-2.23.so
7f9b2a392000-7f9b2a394000 rw-p 00000000 00:00 0 
7f9b2a394000-7f9b2a399000 r-xp 00000000 08:01 399064  /lib/x86_64-linux-gnu/libnss_dns-2.23.so
7f9b2a399000-7f9b2a599000 ---p 00005000 08:01 399064  /lib/x86_64-linux-gnu/libnss_dns-2.23.so
7f9b2a599000-7f9b2a59a000 r--p 00005000 08:01 399064  /lib/x86_64-linux-gnu/libnss_dns-2.23.so
7f9b2a59a000-7f9b2a59b000 rw-p 00006000 08:01 399064  /lib/x86_64-linux-gnu/libnss_dns-2.23.so
7f9b2a59b000-7f9b2a59d000 r-xp 00000000 08:01 399072  /lib/x86_64-linux-gnu/libnss_mdns4_minimal.so.2
7f9b2a59d000-7f9b2a79c000 ---p 00002000 08:01 399072  /lib/x86_64-linux-gnu/libnss_mdns4_minimal.so.2
7f9b2a79c000-7f9b2a79d000 r--p 00001000 08:01 399072  /lib/x86_64-linux-gnu/libnss_mdns4_minimal.so.2
7f9b2a79d000-7f9b2a79e000 rw-p 00002000 08:01 399072  /lib/x86_64-linux-gnu/libnss_mdns4_minimal.so.2
7f9b2a79e000-7f9b2a7a9000 r-xp 00000000 08:01 399066  /lib/x86_64-linux-gnu/libnss_files-2.23.so
7f9b2a7a9000-7f9b2a9a8000 ---p 0000b000 08:01 399066  /lib/x86_64-linux-gnu/libnss_files-2.23.so
7f9b2a9a8000-7f9b2a9a9000 r--p 0000a000 08:01 399066  /lib/x86_64-linux-gnu/libnss_files-2.23.so
7f9b2a9a9000-7f9b2a9aa000 rw-p 0000b000 08:01 399066  /lib/x86_64-linux-gnu/libnss_files-2.23.so
7f9b2a9aa000-7f9b2a9b0000 rw-p 00000000 00:00 0 
7f9b2a9b0000-7f9b2ab70000 r-xp 00000000 08:01 398967  /lib/x86_64-linux-gnu/libc-2.23.so
7f9b2ab70000-7f9b2ad70000 ---p 001c0000 08:01 398967  /lib/x86_64-linux-gnu/libc-2.23.so
7f9b2ad70000-7f9b2ad74000 r--p 001c0000 08:01 398967  /lib/x86_64-linux-gnu/libc-2.23.so
7f9b2ad74000-7f9b2ad76000 rw-p 001c4000 08:01 398967  /lib/x86_64-linux-gnu/libc-2.23.so
7f9b2ad76000-7f9b2ad7a000 rw-p 00000000 00:00 0 
7f9b2ad7a000-7f9b2ada0000 r-xp 00000000 08:01 398939  /lib/x86_64-linux-gnu/ld-2.23.so
7f9b2af81000-7f9b2af84000 rw-p 00000000 00:00 0 
7f9b2af9d000-7f9b2af9f000 rw-p 00000000 00:00 0 
7f9b2af9f000-7f9b2afa0000 r--p 00025000 08:01 398939  /lib/x86_64-linux-gnu/ld-2.23.so
7f9b2afa0000-7f9b2afa1000 rw-p 00026000 08:01 398939  /lib/x86_64-linux-gnu/ld-2.23.so
7f9b2afa1000-7f9b2afa2000 rw-p 00000000 00:00 0 
7ffd0cf9b000-7ffd0cfbc000 rw-p 00000000 00:00 0 [stack]
7ffd0cfc0000-7ffd0cfc2000 r--p 00000000 00:00 0 [vvar]
7ffd0cfc2000-7ffd0cfc4000 r-xp 00000000 00:00 0 [vdso]
ffffffffff600000-ffffffffff601000 r-xp 00000000 00:00 0 [vsyscall]

\end{lstlisting}
}
\noindent
Each row of the information above shows a region of contiguous virtual memory. For example: the first row describes the information:

\begin{table}[!htbp]
\centering
\begin{tabular}{cccccc}
\hline
address & permissions & offset & device & inode & pathname \\
\hline
08048000-0804b000 & r-xp & 00000000 & 08:01 & 281247 & /home/lizhiyuan/Downloads/WebBench/webbench \\
\hline
\end{tabular}
\caption{A line in vm table}
\end{table}

\begin{table}[!htbp]
\centering
\begin{tabular}{m{2cm} m{15cm}}
\hline
address & This is the starting and ending address of the region in the process's address space. \\
\hline
permissions & This describes how pages in the region can be accessed. There are four different permissions: read, write, execute, and shared. If a region is not shared, it is private, so a 'p' will appear instead of an 's'. If the process attempts to access memory in a way that is not permitted, a segmentation fault is generated. \\
\hline
offset & If the region was mapped from a file, this is the offset in the file where the mapping begins.\\
\hline
device & If the region was mapped from a file, this is the major and minor device number (in hex) where the file lives. \\
\hline
inode & If the region was mapped from a file, this is the file number. \\
\hline
pathname & If the region was mapped from a file, this is the name of the file. There are also special regions with names like [heap], [stack], or [vdso]. [vdso] stands for virtual dynamic shared object. It's used by system calls to switch to kernel mode. \\
\hline
\end{tabular}
\caption{Descriptions for VM table fields\cite{vmap}}
\end{table}

\noindent
In this case, we can know from the information shown above that the Webbench binary file is running in the contiguous virtual memory, and there are also shared libraries in the virtual memory.


\subsection{Run strace}
We fistly run the webbench, then we use \textit{ps} command to get the pid, and then we run strace on it.
\begin{lstlisting}
$ ./webbench -f http://abc.go.com/ -t 100
$ ps -a | grep webbench
$ sudo strace -p 4114
\end{lstlisting}
We get the following information.

{
\scriptsize
\begin{lstlisting}
connect(5, {sa_family=AF_INET, sin_port=htons(80), sin_addr=inet_addr("54.235.139.178")}, 16) = 0
write(5, "GET / HTTP/1.0\r\nUser-Agent: WebB"..., 62) = 62
close(5)                                = 0
stat("/etc/resolv.conf", {st_mode=S_IFREG|0644, st_size=191, ...}) = 0
stat("/etc/resolv.conf", {st_mode=S_IFREG|0644, st_size=191, ...}) = 0
open("/etc/hosts", O_RDONLY|O_CLOEXEC)  = 5
fstat(5, {st_mode=S_IFREG|0644, st_size=221, ...}) = 0
read(5, "127.0.0.1\tlocalhost\n127.0.1.1\tub"..., 4096) = 221
read(5, "", 4096)                       = 0
close(5)                                = 0
stat("/etc/resolv.conf", {st_mode=S_IFREG|0644, st_size=191, ...}) = 0
socket(PF_INET, SOCK_DGRAM|SOCK_NONBLOCK, IPPROTO_IP) = 5
connect(5, {sa_family=AF_INET, sin_port=htons(53), sin_addr=inet_addr("127.0.1.1")}, 16) = 0
poll([{fd=5, events=POLLOUT}], 1, 0)    = 1 ([{fd=5, revents=POLLOUT}])
sendto(5, "\207\255\1\0\0\1\0\0\0\0\0\0\3abc\2go\3com\0\0\1\0\1", 28, MSG_NOSIGNAL, NULL, 0) = 28
poll([{fd=5, events=POLLIN}], 1, 5000)  = 1 ([{fd=5, revents=POLLIN}])
ioctl(5, FIONREAD, [114])               = 0
recvfrom(5, "\207\255\201\200\0\1\0\4\0\0\0\0\3abc\2go\3com\0\0\1\0\1\300\f\0\5"..., 1024, 0, {sa_family=
AF_INET, sin_port=htons(53), sin_addr=inet_addr("127.0.1.1")}, [16]) = 114
close(5)                                = 0
socket(PF_INET, SOCK_STREAM, IPPROTO_IP) = 5
connect(5, {sa_family=AF_INET, sin_port=htons(80), sin_addr=inet_addr("50.16.243.56")}, 16) = 0
write(5, "GET / HTTP/1.0\r\nUser-Agent: WebB"..., 62) = 62
close(5)                                = 0
stat("/etc/resolv.conf", {st_mode=S_IFREG|0644, st_size=191, ...}) = 0
stat("/etc/resolv.conf", {st_mode=S_IFREG|0644, st_size=191, ...}) = 0
open("/etc/hosts", O_RDONLY|O_CLOEXEC)  = 5
fstat(5, {st_mode=S_IFREG|0644, st_size=221, ...}) = 0
read(5, "127.0.0.1\tlocalhost\n127.0.1.1\tub"..., 4096) = 221
read(5, "", 4096)                       = 0
close(5)                                = 0
\end{lstlisting}
}
\noindent
The strace command allow us to trace the system calls and signal delivery of the Webbench process. These information only shows partial output. We can see some functions are running, such as recvfrom(), socket(), connect(), read(), write(), close(). Because the Webbench is a tool for benchmarking WWW or proxy servers, it will use fork() for stimulating multiple clients. So in the screenshot there are a lot of functions like recvfrom(), socket(), read(), and write().

\subsection{About WebBench}
WebBench is a tool for testing the performance of servers under heavy loads. It uses the fork() system call for simulating multiple clients at a time and can let us know if a particular system is capable of handling several client without pulling down the system. \\

\noindent
The main functionality of webbench.c is to benchmarking WWW or proxy servers. In the webbench.c source code, it allows users to use different options to benchmark specific URL by simulating multiple clients to connect to the server and also record the successful connection times as well as failed connection times. \\

\noindent
For example, we can use WebBench as shown in the screenshot below:
\begin{figure}[!htbp]
\centering
\includegraphics[height=7.5cm]{WebBenchExample.png}
\caption{Image showing the source code folder after running make command}
\end{figure}

\noindent
The above screenshot explains that for the website http://youtube.com/, we simulated 500 clients for 30 seconds which resulted for a parsing speed of 3360 pages/min with data communication rate of 7398 bytes/sec. Out of the total 1680 resquests bombarded at the server, 1666 completed successfully and 14 failed.

\subsection{WebBench Fuctionalities}
\noindent
Below are few of the options provided by webbench:
\begin{center}
\begin{tabular}{ |r|r|m{13cm}|} 
     \hline
     1 & webbench -r URL & -r allows us to send reload requests to the user instead of the normal hits.\\ 
     \hline
     2 & webbench -t URL & -t allows us to limit/extend the time limit for which the test has to be performed.Default is 30 seconds.\\ 
     \hline
     3 & webbench -p URL & -p allows us to use a proxy server for sending requests.\\ 
     \hline
     4 & webbench -c URL & -c allows us to run `n' number of HTTP clients at once. Default is 1.\\
     \hline 
     5 & webbench -9 URL & -9 gives us the flexibility to use HTTP/0.9 style requests.\\
     \hline
     6 & webbench -1 URL & -1 gives us the flexibility to use HTTP/1.0 style requests.\\
     \hline 
     7 & webbench -1 URL & -2 gives us the flexibility to use HTTP/1.1 style requests.\\
     \hline
\end{tabular}
\end{center}

\noindent
Below are system-level functions of webbench:
\begin{center}
\begin{tabular}{ |l|m{13cm}|} 
\hline
main() & This is the function that allow the entry to the program. In main(), there is user interface that provides user to read the different options of and choose different options and also input the URL. \\ 
\hline
bench() & Create a specific number of clients in order to connect to the server. The maximum number that application can create is 30,000. \\ 
\hline
usage() & Print the options and information about each option. \\
\hline
build\_request() & According to the URL the user entered, this function verify the validity of that URL. \\
\hline 
benchcore() & Control the benchmarking process and record the successful or fail connections. \\
\hline
Socket() & Define the socket that provides for client to connect to server.\\

\hline
\end{tabular}
\end{center}

\subsection{System Architectural Diagram}
\begin{figure}[!htbp]
\centering
\includegraphics[height=5cm]{sa.jpeg}
\caption{System Architecture}
\end{figure}

\section{From \textit{SciTools Understand}}
\subsection{Import the source into \textit{SciTools Understand}}
We modify the Makefile of the software. Now the Makefile is:
{
\scriptsize
\begin{lstlisting}
CFLAGS?=	-Wall -ggdb -W -O
CC?=    gccwrapper
LIBS?=
LDFLAGS?=
PREFIX?=	/usr/local/webbench
VERSION=1.5
TMPDIR=/tmp/webbench-$(VERSION)

all:   webbench tags

tags:  *.c
	-ctags *.c

install: webbench
	install -d $(DESTDIR)$(PREFIX)/bin
	install -s webbench $(DESTDIR)$(PREFIX)/bin
	ln -sf $(DESTDIR)$(PREFIX)/bin/webbench $(DESTDIR)/usr/local/bin/webbench

	install -d $(DESTDIR)/usr/local/man/man1
	install -d $(DESTDIR)$(PREFIX)/man/man1
	install -m 644 webbench.1 $(DESTDIR)$(PREFIX)/man/man1
	ln -sf $(DESTDIR)$(PREFIX)/man/man1/webbench.1 $(DESTDIR)/usr/local/man/man1/webbench.1

	install -d $(DESTDIR)$(PREFIX)/share/doc/webbench
	install -m 644 debian/copyright $(DESTDIR)$(PREFIX)/share/doc/webbench
	install -m 644 debian/changelog $(DESTDIR)$(PREFIX)/share/doc/webbench

webbench: webbench.o Makefile
	$(CC) $(CFLAGS) $(LDFLAGS) -o webbench webbench.o $(LIBS) 

clean:
	-rm -f *.o webbench *~ core *.core tags
	
tar:   clean
	-debian/rules clean
	rm -rf $(TMPDIR)
	install -d $(TMPDIR)
	cp -p Makefile webbench.c socket.c webbench.1 $(TMPDIR)
	install -d $(TMPDIR)/debian
	-cp -p debian/* $(TMPDIR)/debian
	ln -sf debian/copyright $(TMPDIR)/COPYRIGHT
	ln -sf debian/changelog $(TMPDIR)/ChangeLog
	-cd $(TMPDIR) && cd .. && tar cozf webbench-$(VERSION).tar.gz webbench-$(VERSION)

webbench.o:	webbench.c socket.c Makefile

.PHONY: clean install all tar
\end{lstlisting}
}
\noindent
And then we use the \textit{buildspy} command to build the binary and
simultaneously create an understand database.
\begin{lstlisting}
buildspy -db ~/Documents/sva_hw2.udb -cmd "make CC=gccwrapper"
\end{lstlisting}
Then we use \textit{Understand} open the .udb file. And then analyze all files.

\subsection{Plot the metrics summary}
Below is the Summary Metrics, which we were able to extract from the Understand tool. The Metrics tells a lot of details about the code for WebBench, which give us a broaden understanding of the WebBench source code. The metrics is a very important of our activity because it tells us what part of our code is dedicated to what. For example, the metrics says that the WebBench code has only 2 C files containing 7 Functions and total of 552 lines of code. It also provide some of the valuable details like the Comment to Code ratio, which can provide us with some basic information of the project.
\begin{table}[!htbp]
\centering
\begin{tabular}{r c}
\hline
Blank Lines & 67 \\
\hline
Classes & 0 \\
\hline
Code Lines & 367 \\
\hline
Comment Lines & 95 \\
\hline
Comment to Code Ratio & 0.26 \\
\hline
Declarative Statements & 42 \\
\hline
Executable Statements & 249 \\
\hline
Files & 2 \\
\hline
Functions & 7 \\
\hline
Inactive Lines & 0 \\
\hline
Lines & 552 \\
\hline
Preprocessor Lines & 26 \\
\hline
\end{tabular}
\caption{Metrics}    
\end{table}


\subsection{Cluster Call Butterfly Graph}
Below are the Cluster Call Butterfly Graph of few of the methods of WebBench. Cluster Call Butterfly method inform us about the relations of different methods inside a C file and also relation of a C function with any other C file. For example, in the below figure we can see that the method bench of C file webbench.c calls benchcore method of the same C file (webbench.c). Also, it is also showing that the method bench is calling a different C class called socket.c. It gives us a basic understanding about what the project cluster call like.

\begin{figure}[!htbp]
\centering
\includegraphics[height=5cm]{cluster_call.png}
\caption{Cluster Call Butterfly Graph}
\end{figure}


\subsection{Control flow graph}
We choose the \textit{main()}, \textit{bench()} and \textit{benchcore()} functions to generate control flow graphs. \\
A control flow diagram provides us with the details about the flow of the logic in the code. Here, we have generated the control flow diagram of the whole project of WebBench (on the main function) and it depicts the flow of logic of the complete project. It is very helpful for the people doing reverse engineering as it provides a quick review of the logic flow from one method to the other. Just by taking a glace at the graph one can easily deduce about the function’s or class call.

\begin{figure}[!htbp]
\centering
\includegraphics[height=19cm]{cf_main.png}
\caption{Control Flow for \textit{main()}}
\end{figure}

\clearpage

\begin{figure}[!htbp]
\centering
\includegraphics[height=22cm]{cf_bench.png}
\caption{Control Flow for \textit{bench()}}
\end{figure}

\clearpage

\begin{figure}[!htbp]
\centering
\includegraphics[height=19cm]{cf_benchcore.png}
\caption{Control Flow for \textit{benchcore()}}
\end{figure}


\subsection{Identifying \textit{main()} function vulnerabilities}
The main function of WebBench takes URL as a user provided input. The method accepts the parameter as a plain text and the method does not employ any method to encrypt/decrypt the URL. This itself is one of the major threats for the software. An attacker can flood the application with huge amount of traffic and can overload the application at any given time. Also, an unencrypted URL is highly vulnerable to URL manipulation and thus can depict some serious behavior in the system. 

\noindent
In the main method we came across 2 variables (optarg and clients) which are directly dependent on the input parameter and they are very much available for any buffer overflow exploit as they are using the format specifier for storing the inputs. The attacker can pass his own parameter by overriding the main method. It may be tough for him in the beginning to guess the behaviour, but with few random inputs he will be able to guess it. Also, we got another vulnerability where the input argument does not warn the user for the limit of number of arguments. While providing the inputs we came to know that there is no check provided by the rogramme on the number of parameters that it can take but it only processes the first input provided in the sequence of parameters. 

\noindent Most of the vulnerabilities are discovered due to some unwanted behaviors in the system produced by inefficient code. However, any inefficient system does not start demonstrating such behaviors from the first day, the system has to wait for a powerful enough adversary to take over the capability of any functionality. So, it wont be incorrect if we assume that most of the vulnerabilities are product of poor implementation of code. The people who discover these kind of behaviors have sound knowledge of implementation of accurate solution of the requirement.

\subsection{Aim for finding the vulnerabilities}
While going through the code for identifying any potential vulnerability, our purpose is to identify the locations in code where the programmer has inefficiently crafted the logic. By inefficient logic, we mean that there can be a situation where the programmer was unaware of the actual behaviour of the logic which he/she implemented. This can cause him to implement the logic in such a way that can be an open invitation for the attacker. 

While looking for vulnerabilities, we focus on several factors of the programme such as encryption of input, proper initialization of variables and also their scope. We also focus on whether there is any loop-hole in handling of variables when it is passed from one method to another. In short, we take care of the complete life-cycle of any variable so that it does not get misused by the attacker.

\section{Reverse engineering ELF files}
\subsection{Plot the entropy}
We firstly use \textit{http://binvis.io/} to generate entropy of webbench. Entropy of a binary file provide us with a detail about the degree of disorder which it possess. If we have a data set where all the elements have the same value ,the amount of disorder will be nil and the entropy will be zero. If a data set has maximum amount of heteroginity,it is said to contain maximum entropy. Reverse enginnering mainly focuses on high order of entropy as the concerned data firstly provide them with the information about the compressed data and cryptographic data.\cite{plot_ent}

\begin{figure}[!htbp]
    \centering
    \includegraphics[height=10cm]{Entropy.png}
    \caption{Entropy of binary file of webbench}
\end{figure}

\newpage
\noindent
From the above figure we can identify that most part of the image is showing black colours which means that the entropy is zero in those regions. With zero entropy we should understand that the data at those locations are not heterogenous. On the other hand, in the region of blue shade, they have intermediate entropy, such entropy regions depict normal texts and data whose re-occurances is mild. There are very few pink regions, which tells us about the regions with high entropy. High entropy is a region which shows data, the occurance of which is mostly once and it is highly encrypted. So, the more pink is the entropy image, the more it is secured, compressed or encrypted.

\subsection{Reverse Engineering a portion of the code}
\begin{lstlisting}
    void benchcore(const char *host,const int port,const char *req)
    {
        int rlen;
        char buf[1500];
        int s,i;
        struct sigaction sa;
    
        /* setup alarm signal handler */
        sa.sa_handler=alarm_handler;
        sa.sa_flags=0;
        if(sigaction(SIGALRM,&sa,NULL))
            exit(3);
        
        alarm(benchtime); // after benchtime,then exit
    
        rlen=strlen(req);
        nexttry:while(1)
        {
            if(timerexpired)
            {
                if(failed>0)
                {
                    /* fprintf(stderr,"Correcting failed by signal\n"); */
                    failed--;
                }
                return;
            }
            
            s=Socket(host,port);                          
            if(s<0) { failed++;continue;} 
            if(rlen!=write(s,req,rlen)) {failed++;close(s);continue;}
            if(http10==0) 
            if(shutdown(s,1)) { failed++;close(s);continue;}
            if(force==0) 
            {
                /* read all available data from socket */
                while(1)
                {
                    if(timerexpired) break; 
                    i=read(s,buf,1500);
                    /* fprintf(stderr,"%d\n",i); */
                    if(i<0) 
                    { 
                        failed++;
                        close(s);
                        goto nexttry;
                    }
                    else
                    if(i==0) break;
                    else
                    bytes+=i;
                }
            }
            if(close(s)) {failed++;continue;}
            speed++;
        }
    }
}
\end{lstlisting}

\noindent
The above method  takes the input as the hostname and port and the request pointer location. Below are some of the fields which are used in the above programme:
\begin{center}
    \begin{tabular}{ |c|r|m{13cm}|} 
         \hline
         1 & rlen & it keeps the length of the requests obtained through the parameter. It takes care of all the requests which are getting send by verifying if the testing has been done on it or not.\\ 
         \hline
         2 & timerexpired  & it is a parameter which tells the programme to test for only those many seconds. Once the timeexpired is reached the testing is closed for that particular request.\\ 
         \hline
         3 & failed & failed is a counter which keeps a track of how many requests failed out of a certain number of hits to the server.\\ 
         \hline
         4 & benchtime & benchtime is responsible for setting up an alarm which tells the programm to exit after the provided time.\\
         \hline 
\end{tabular}
\end{center}

\subsection{The Python script}
For read the ELF file and getting some information out of it, we created a python script. 
Following is the script:
{
\tiny
\begin{lstlisting}[language=Python]
def parser():
    file = open('webbench', 'rb')
    
    file.seek(0x0, 0)
    str1 = file.read(4)
    file.seek(0x1F68, 0)
    str2 = file.read(8)
    file.seek(0x609c, 0)
    str3 = readSkip(file.read(348))

    print(str1.decode('ascii'))
    print(str2.decode('ascii'))
    print(str3)
    #print(int(str3, 16))

def readSkip(buffer):
    ba = [hex(x) for x in buffer]
    strs = ''
    for x in ba:
        if int(x, 16) <= 0xff:
            
            if int(x, 16) == 0x1 or int(x, 16) == 0x2 or int(x, 16) == 0x3 or int(x, 16) == 0x4 or int(x, 16) == 0x5:
                continue
            strs += chr(int(x, 16))
    return strs;

if __name__ == '__main__':
    parser()
\end{lstlisting}
}
\noindent
When running the script, we get the file type, file name, and some including information like socket.c, string3.h, etc. Here is the screenshot.

\begin{figure}[!htbp]
\includegraphics[height=1.3cm]{script.jpg}
\caption{Script Execution Results}
\end{figure}

\begin{thebibliography}{10}
\bibitem{objdump}
L. File), \"Linux Objdump Command Examples (Disassemble a Binary File)\", Thegeekstuff.com. [Online]. Available: http://www.thegeekstuff.com/2012/09/objdump-examples/?utm\_source=feedburner.

\bibitem{vmap}
U. /proc/id/maps, \"Understanding Linux /proc/id/maps\", Stackoverflow.com. [Online]. Available: https://stackoverflow.com/questions/1401359/understanding-linux-proc-id-maps.

\bibitem{plot_ent}
\"cortesi - Visualizing entropy in binary files\", Corte.si. [Online]. Available: https://corte.si/posts/visualisation/entropy/index.html.
\end{thebibliography}
\end{document}