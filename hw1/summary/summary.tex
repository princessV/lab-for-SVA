\documentclass[journal]{IEEEtran}


% ------------ package ------------
\usepackage{graphicx}
\graphicspath{{public/}}
% ------------ package end ------------

\hyphenation{op-tical net-works semi-conduc-tor}


\begin{document}

% ------------ title ------------
\title{Threat Modeling for SQLite}
\author{Qiyang~Gu, Ujjawal~Sharma, Zhiyuan~Li}
\maketitle
% ------------ title end ------------


% ------------ abstract ------------
\begin{abstract}
 Threat modeling is a technique meant to build secure systems by searching for the potential threats in systems.\cite{cite_wiki_threat_modeling} The concepts of threat model are based on assets which are worth being protected. Many a times, while validating a functionality of product, we come across some of the assets which have specific vulnerabilities which can be exploited by attackers. Security analysts can identify which component in a system that would have the highest risk of being attacked by thinking from the perspective of the attacker. These components should have highest priorities in security assessments. The more complex a product,the more rigorous and thorough should be the analysis. The importance of threat modelling is more when we are dealing with a product which is responsible for handling crucial information. The best example of such a product can be any database. A Database is an essential and very significant part in an enterprise. It contains pivotal information about whole system and always become the target of attackers. In this report we choose on open - source database engine SQLite, following the threat modeling process of Microsoft,and created a threat model for SQLite.\cite{cite_wiki_threat_modeling}
\end{abstract}
% ------------ abstarct end ------------

% ------------ keywords ------------
\begin{IEEEkeywords}
sqlite3, database, threat modeling
\end{IEEEkeywords}
% ------------ keywords end ------------



\section{Introduction to sqlite}
\IEEEPARstart{S}QLite is a lightweight open source SQL database engine, written totally in C Programming Language. It is self-contained, serverless, zero-configuration and transcational. It takes up very few resources when running. Not only it can support different operating systems like Windows/Linux/Unix and also for Android, but also it can be easily integrated with different programming languages like Java/PHP/C\#. These features are enough to make it a versatile product in itself. Also, when compared to MySQL and PostgreSQL, the world's most loved open-source database engines to developers, SQLite has a better performance on query as well. It has been 17 years since the alpha version of SQLite released in May 2000. Now, SQLite3 is pre-installed in almost all the Linux systems, as well as the latest macOS.\cite{cite_sqlite_home_page}






\section{Reasons on sqlite}
As the use of SQLite is increasing everyday, it has become important to keep an eye on what issues the consumers can face in case of any attack. One of the main reason for selecting SQLite is because it is getting widely used in electronic devices and robots. As IoT is expanding its tentacles everyday in one or the other form, it has itself become an important topic of study. A study here means the behavior the product based on IoT can demostrate in case of some adversaries. Almost most of the electronic devices around us can have the SQLite implemented in it. It even has good performance as compared to other engines in network connectivity operations. Apart from the few mentioned above, below are some other famous applications of SQLite:

\begin{itemize}
\item As a File-Format database: There are many devices/softwares which needs there database to be kept as a file system. SQLite's journal provides a good option here.These softwares may be catalog files for e-Commerce or record keeping websites.
\item Lightweight and moderate traffic websites: Developers who have relatively low requirement in terms of traffic and content, love to use such lightweight databases.
\item Websites handling large datasets: When the database engine is lightweight, it is able to handle large datasets efficiently.
\item As a cache: Many web applications, for reducing latency use SQLite. It gives the developer a flexibility to make their application to operate in a local environment rather than using the network connection to connect to the main database and it could also run in case of network outages(offline state).
\end{itemize}

\section{Goals}
Before we start, we set up our goals on this projects. So that we can focus on what we need to do. These goals inclues,
\begin{itemize}
\item To explore how database engine works (not only SQLite, but also other database engines like MySQL, PostgreSQL) and what is the architecture of a database engine. To know the advantadges and disadvantadges of SQLite.
\item To figure out what vulnerabilities may exist in SQLite.
\item To find ways in which we may fix these vulnerabilities.
\item To learn what is threat modeling, how threat modeling works and why we need to do threat modeling.
\item To generate reports for threat modeling. These reports can be references for other threat modeling work in the future.
\end{itemize}


\section{Project execution highlights}
Based on our goals, below are things we primarily do in the project.
\begin{itemize}
\item Installed SQLite, do feature tests on Linux and macOS. Figured out what it provides to users and how it works.
\item Downloaded the source code from offical website. Tried to figure out the structure, main function of the source codes. Tried to draw a architecture diagram.
\item Followed the homework instructions to do threat modeling on SQLite, including
\begin{itemize}
\item Identify assets. On this step, we find the only asset of sqlite.
\item Create an architecture overview. On this step, we use Microsoft Threat Modeling Tool to analysis SQLite based on our understandings (like the functions, modules.)
\item Decompose the application. On this step, we refine our high-level threat modeling and discuss the questions in security profile.
\item Identify the threats, document and rate these threats.
\end{itemize}
\end{itemize}




\section{Accomplished \& Unaccomplished}
\subsection{Accomplished}



\subsection{Unaccomplished}







\section{Conclusion}
SQLite is a popular open-source database engine. It is very lightweight meanwhile its performance is great. Many small and medium-sized websites prefer to use SQLite as their database. However, it's lightweight feature also brings many security issues. We cannot expect such a small database engine offers great secure policies. \\
During the process of threat modeling, we identify the assets, decomposing the whole application, identifying the threats and valuing these threats. We also come up with some mitigations for these threads. \\ 
It's not easy work for our green-hand students. However, after finishing the whole assignment, we have learnt how to consider more detailedly when building a software. 

\section*{Acknowledgment}
We would like to thank Jay for his hints and instructions to do the project. We once got some misunderstanding on the part of the project, which makes us somewhat \'uncomfortable\' to do the project. His passion on the work does touch us. We are sorry maybe we still do not finish the project very well because of too many dues these days, but we will try our best to work on homework in the future.

\begin{thebibliography}{10}
\bibitem{cite_wiki_threat_modeling}
\"Threat model\", En.wikipedia.org. [Online]. Available: https://en.wikipedia.org/wiki/Threat\_model.
\bibitem{cite_book_threat_modeling}
2014, Shostack, Threat Modeling: Designing for Security
\bibitem{cite_sqlite_home_page}
\"SQLite Home Page\", Sqlite.org. [Online]. Available: https://www.sqlite.org/index.html.
\bibitem{cite_source_code_analysis}
\"SQlite source code analysis\", Huili.github.io. [Online]. Available: http://huili.github.io/.
\end{thebibliography}

\end{document}