\documentclass[a4paper, 11pt]{article}

\usepackage{graphicx}
\graphicspath{{public/}}

\usepackage{geometry}
\geometry{left=1.5cm, right=1.5cm, top=3cm, bottom=2cm}

\usepackage{array}
\newcommand{\tabincell}[2]{\begin{tabular}{@{}#1@{}}#2\end{tabular}}



\begin{document}

\title{\textbf{Appendix}}
\author{\textit{Qiyang Gu, Ujjawal Sharma, Zhiyuan Li}}
\date{}
\maketitle


\section*{Step 1: Identify assets}
An asset is defined as a conceptual or concrete property which is of attacker’s importance. They can be termed as threat targets. In our case, the only thing which SQlite \'wants to protect\' is the database file. Every SQL command executed will finally change the single database file. The file is the only source which SQLite reads from the local disk. Therefore, it's the only asset and is extremely significant.


\begin{table}[htbp]

\centering
\begin{tabular}{ m{1cm}  >{\raggedright}m{4cm}  m{10cm} }    
\hline
ID & Name & Description \\
\hline
1 & Single Database File & The single database file contains all the information the database engine generates, including tables, views, triggers, indexes, database backup and transaction log. \\
\hline
\end{tabular}
\caption{Assets of SQLite}
\end{table}


\section*{Step 2: Create an architecture overview}
\setcounter{section}{2}
\subsection{Functional Decomposition of SQLite}

\begin{table}[!htbp]
\centering
\begin{tabular}{ll}
\hline
Function & Description \\
\hline
sqlite3\_open() & To create or open a database \\
\hline
sqlite3\_exec() & To execute the SQL query or the SQL command \\
\hline
sqlite3Prepare() & To preprocess the SQL code \\
\hline
sqlite3Parser() & To parse the SQL code \\
\hline
Tokenizer() & To analyze the SQL code \\
\hline
codeGenerator() & To generate VDBE byte code(operators) \\
\hline
sqlite3\_step() & To initialize the process of executing the operators \\
\hline
sqlite3VdbeCreate() & To create a VDBE \\
\hline
sqlite3VdbeRewin() & To initialize a VDBE \\
\hline
sqlite3VdbeMakeRead() & To prepare the executing process \\
\hline
sqlite3VdbeExec() & To execute the operators from SQL statement compiler \\
\hline
sqlite3VdbeFreeCursor() & To turn off the VDBE cursor and release all resources used by the cursor \\
\hline
closeAllCursors() & To turn off all the cursors and release any occupied VDBE in dynamic memory \\
\hline
sqlite3VdbeFinalize() & To close the VDBE when the operations are completed \\
\hline
sqlite3VdbeDelete() & To delete the VDBE \\
\hline
sqlite3\_close() & To close SQLite \\
\hline
\end{tabular}
\caption{Functional Decomposition of SQLite}
\end{table}


\subsection{Technologies}
\clearpage
\begin{table}[!htbp]
\centering
\begin{tabular}{>{\raggedright}m{2cm} m{10cm}}
\hline
Technology & Description \\
\hline
B-Tree & B-tree is a self-balancing tree data structure that keeps data sorted and allows searches, sequential access, insertions, and deletions in logarithmic time. \\
\hline
R-Tree & R-tree is a special index for the row of a query design. R-tree generally querier minimum or maximum rectangle in the xy coordinate GIS. \\
\hline
Tokenizer and Parser & The Tokenizer and Parser process the SQL statements and SQLite command, and convert them to SQLite internal byte codes that can be executed by Virtual Database Engine. \\
\hline
Page Cache & Because the available memory is limited, only a portion of the memory is used to store data from the database files. This part of the memory space is usually called a database cache or data buffer. In SQLite, it is called a page cache. \\
\hline
C Programming Language & C is a general-purpose, imperative computer programming language, supporting structured programming, lexical variable scope and recursion, while a static type system prevents many unintended operations. \\
\hline
\end{tabular}
\caption{Technologies used in SQLite}
\end{table}

\subsection{Architecture diagram}
\begin{figure}[!htbp]
\centering
\includegraphics[height=7.7cm]{sqlite1.png}
\caption{Architecture of SQLite}
\end{figure}

\vspace{1cm}

\section*{Step 3: Decompose the application}
\setcounter{section}{3}
\setcounter{subsection}{0}
\subsection{Security Profiles}
\clearpage
\begin{table}[!htbp]
\centering
\begin{tabular}{>{\raggedright}m{3cm} m{13cm}}
\hline
Input Validation & \tabincell{m{13cm}}{1. Is all input data validated? \\ Not all the input is validated, users can type any SQL statements or other command to the front-end interface. Some invalid statements will be omitted by Tokenizer. Other statements, which may be malicious, will be executed. \\ 2. Could an attacker inject commands or malicious data into the application? \\ Yes. The attacker can conduct SQL injection into the application that based on SQLite \\ 3.	Is data validated as it is passed between separate trust boundaries (by the recipient entry point)? \\ Not all the data is validated as it is passed between separate trust bound. \\ 4. Can data in the database be trusted? \\ All the data in the database is created by user/program, so not all the data can be trusted.} \\
\hline
Authentication & \tabincell{m{13cm}}{1. Are strong account policies used? \\ SQLite runs only on local machine. Therefore, there are no strong account policies, no passwords enforced, no certificates, no password verifiers.} \\
\hline
Authorization & SQLite does not have a user management system, which means every user can get access to the whole database file without authorization. \\
\hline
Configuration management & SQLite provides a set of APIs for users to configure database. These APIs can be called everywhere by any user. There are no specific security policies. \\
\hline
Sensitive data & \tabincell{m{13cm}}{1. All the SQL statements and database command are sensitive in the application. \\ 2. The user can get access to the whole database file. There are no specific secure policy to protect the file. And no encryption is enabled.} \\
\hline
Session management & Since no network are enabled, there is no session management. \\
\hline
Cryptography & The SQlite database is not responsible for encrypt the sensitive data and the data will be stored in plaintext. So there are no encryption algorithm in the SQLite database. \\
\hline
Parameter manipulation & \tabincell{m{13cm}}{1. The SQLite database itself does not detect tampered parameters since the applications that use SQLite will be responsible for detecting tampered parameters, such as a Web portal should implement the function to detect tampered parameters. \\ 2. The SQLite database itself does not validate all parameters in form fields, view state, cookie data, and HTTP headers.} \\
\hline
Exception management & \tabincell{m{13cm}}{1. When the developers implemented the SQLite, they considered most of possible exceptions in the C source code. \\ 2. Also, when the exception occurs, the SQLite database user can check the log files, such as the transaction logs to identify the exceptions. \\ 3. The exceptions are allowed to propagate back to the users since the log files are stored in the disk. \\ 4. The exception information will be recorded in the log files, which contain the exploitable information.} \\
\hline
Auditing and logging & \tabincell{m{13cm}}{1. The SQLite database stores log files(database files) in disk, which means that the attacks can be able to access these files by copying the disk of the system. \\ 2. Unlike most of database engines are client/server based, the SQLite is serverless. Any program that is able to access the disk is able to use the SQLite database.} \\
\hline
\end{tabular}
\caption{Security profiles in SQLite}
\end{table}



\setcounter{subsection}{1}
\subsection{Trust Boundaries, Data Flow and Entry Point}
\clearpage
\begin{figure}[!htbp]
\centering
\includegraphics[height=7.7cm]{sqlite2.png}
\caption{Trust Boundaries of SQLite}
\end{figure}

\vspace{1cm}

\section*{Step 4: Identify the threats}
\setcounter{section}{4}
\setcounter{subsection}{0}
\subsection{STRIDE}
\subsubsection{Spoofing}
\begin{table}[!htbp]
\centering
\begin{tabular}{| m{2.5cm} | m{4cm} | m{5cm} | m{4cm} |}
\hline
Property & Threat Definition & Examples & Mitigations \\
\hline
Authentication & Spoofing is considered as the acquiring an illegal access to someone's account by acquiring username, password and other authenticating information. An attacker places himself in place of the original user so as to gain his capabilities in the application. & \tabincell{m{5cm}}{1. An attacker can try to authenticate into the database as an admin user and can create serious configuration changes which could make the whole database inaccessible to anyone in future. \\ 2. An attacker can create a web application of their own which will communicate with the databse in a way to fetch important data illegally using original user's authenticating information.} & SQLite currently uses the Ceaser-Cypher authentication (SALT Based) technique which is not considered to be as a strong encryption method. We may implement the RADIUS protocol or a two-step authentication method such as Token Cards. \\
\hline
\end{tabular}
\caption{Spoofing}
\end{table}

\subsubsection{Tampering}
\clearpage
\begin{table}[!htbp]
\centering
\begin{tabular}{| m{2.5cm} | m{4cm} | m{5cm} | m{4cm} |}
\hline
Property & Threat Definition & Examples & Mitigations \\
\hline
Integrity & Tampering is known as a malicious act of modifying the already present data in the database or modifying a flowing data over the network. & An attacker can get an illegal access to the database and can modify the persistant data according to his will. & \tabincell{m{4cm}}{1. SQLite writes the data to a journal file for which there should be strong access control policies. \\ 2. Encryption of code that has the logic for communicating with databse.} \\
\hline
\end{tabular}
\caption{Tampering}
\end{table}


\subsubsection{Repudiation}
\begin{table}[!htbp]
\centering
\begin{tabular}{| m{2.5cm} | m{4cm} | m{5cm} | m{4cm} |}
\hline
Property & Threat Definition & Examples & Mitigations \\
\hline
Non-Repudiation & Repudiation is a situation where the user can deny performing a system transaction on his side and there is no way to track it in case of absence of proper audit entries. & \tabincell{m{5cm}}{1. A user can deny that he performed a particular transaction on a particular date. \\ 2. A user can deny reception of any important mail from the system.} & \tabincell{m{4cm}}{1. Implementing strong column constraints on each table. \\ 2. Auditing each and every event/transaction taking place in the modification of table data or table design.} \\
\hline
\end{tabular}
\caption{Repudiation}
\end{table}


\subsubsection{Information Disclosure}

\begin{table}[!htbp]
\centering
\begin{tabular}{| m{2.5cm} | m{4cm} | m{5cm} | m{4cm} |}
\hline
Property & Threat Definition & Examples & Mitigations \\
\hline
Confidentiality & Information disclosure is exposing of important information to a user who should not have the access to view it. & A normal user should not have the access to edit or delete the information of any table directly which is mostly the job of a Database administrator. & Placing proper access based protocols such as DAC, MAC, RBAC and ABAC. SQLite uses journal to store the data, so, in order to provide a complete protection, implementing the protocols on the journal files is also important. \\
\hline
\end{tabular}
\caption{Information Disclosure}
\end{table}

\subsubsection{Denial of Service}
\clearpage
\begin{table}[!htbp]
\centering
\begin{tabular}{| m{2.5cm} | m{4cm} | m{5cm} | m{4cm} |}
\hline
Property & Threat Definition & Examples & Mitigations \\
\hline
Denial of Service & It is a sitation when the attacker performs such an activity which makes the system unavailable or unusable to the valid users. & \tabincell{m{5cm}}{1. Obstructing the traffic from user so that his requests are not able to communicate with the database. \\ 2. Diverting the request traffic from users to some other webpage.} & Preventing traffic coming from unknown sources by implementing Firewall securities. \\
\hline
\end{tabular}
\caption{Denial of Service}
\end{table}


\subsection{Attack Tree}
\begin{figure}[!htbp]
\centering
\includegraphics[height=8cm]{attack_tree}
\caption{Attack Tree}
\end{figure}
\vspace{1cm}



\section*{Step 5: Document And Rate the Threats}
\begin{table}[!htbp]
\centering
\begin{tabular}{m{4cm} m{11cm}}
\hline
Threat Description & Attacker acquiring the user credentials of the original user. \\
\hline
Threat Target & To modify the persistent data in database. \\
\hline
Risk & High \\
\hline
Attack Techniques & Phishing Techniques, Decrypting URL \\
\hline
Countermeasures & Using HTTPS authentication. Monitor origin of traffic. \\
\hline
\end{tabular}
\caption{Threat Rate}
\end{table}

\begin{table}[!htbp]
\centering
\begin{tabular}{m{4cm} m{11cm}}
\hline
Threat Description & Attacker gets an illegal access to database for modifying data. \\
\hline
Threat Target & Database tables \\
\hline
Risk & High \\
\hline
Attack Techniques & URL manipulation, architecture flaws \\
\hline
Countermeasures & Implementing strong column constraints, checksum on each table and URL encryptions. \\
\hline
\end{tabular}
\caption{Threat Rate}
\end{table}

\begin{table}[!htbp]
\centering
\begin{tabular}{m{4cm} m{11cm}}
\hline
Threat Description & User denying performing a transaction. \\
\hline
Threat Target & Data change done by mistake having no roll - back legally. \\
\hline
Risk & Medium \\
\hline
Attack Techniques & Cryptanalysis and Brute Force Decryption \\
\hline
Countermeasures & Strong encryption of code that handles SQL queries. \\
\hline
\end{tabular}
\caption{Threat Rate}
\end{table}

\begin{table}[!htbp]
\centering
\begin{tabular}{m{4cm} m{11cm}}
\hline
Threat Description & Incorrect user getting access to important data. \\
\hline
Threat Target & No target, result of poor design. \\
\hline
Risk & Low \\
\hline
Attack Techniques & N/A \\
\hline
Countermeasures & Inspecting the access protocols. \\
\hline
\end{tabular}
\caption{Threat Rate}
\end{table}

\begin{table}[!htbp]
\centering
\begin{tabular}{m{4cm} m{11cm}}
\hline
Threat Description & Attacker diverts/obstructs the user traffic from reaching the application, thus, demonstrating poor performance of system. \\
\hline
Threat Target & User credentials and reducing the faith in system performance. \\
\hline
Risk & Medium \\
\hline
Attack Techniques & DNS Spoofing \\
\hline
Countermeasures & Using HTTPS authentication. Monitor origin of traffic. \\
\hline
\end{tabular}
\caption{Threat Rate}
\end{table}

\end{document}