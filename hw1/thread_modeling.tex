\documentclass[a4paper, 11pt]{article}

\usepackage{graphicx}
\graphicspath{{public/}}

\usepackage{geometry}
\geometry{left=1.5cm, right=1.5cm, top=3cm, bottom=2cm}

\usepackage{array}
\newcommand{\tabincell}[2]{\begin{tabular}{@{}#1@{}}#2\end{tabular}}



\begin{document}

\title{\textbf{Threat modeling deliverables}}
\author{\textit{Qiyang Gu, Zhiyuan Li}}
\date{}
\maketitle


\section*{Step 1: Identify assets}
An asset is defined as a conceptual or concrete property which is of attacker’s importance. They can be termed as threat targets. In our case, the SQLite Database Engine, all the non-null attributes of extracted database are considered as very important assets. 

\begin{table}[htbp]

\centering
\begin{tabular}{| m{1cm} || >{\raggedright}m{4cm} || m{10cm} |}    
\hline
\multicolumn{1}{|c||}{ID} & \multicolumn{1}{c||}{Name} & \multicolumn{1}{c|}{Description} \\
\hline
1 & System & Assets relating to the underlying system. \\
\hline
1.1 & Availability of SQLite Database & The SQLite Database should be available 24 hours a day and can be accessed by all database users, database administrators, and system administrators. \\
\hline
1.2 & Ability to Execute SQL Command as a Database Read User & This is the ability to execute SQL select queries on the database, and thus retrieve any information stored in database. \\
\hline
1.3 & Ability to Execute SQL as a Database Read/Write User & This is the ability to execute SQL command, such as select, insert, and update queries on the database and thus have read and write access to any information stored in database. \\
\hline
2 & Database Engine & Assets relating to the SQLite database engine. \\
\hline
2.1 & Login Session & This is the login session of a user to the Sqlite database. This user could be a normal user, database administrator, or a system administrator. \\
\hline
2.2 & Data Stored in Database & The information is stored in SQLite database in various tables using different attributes, which may include sensitive information such as username, password, email address, banking details. \\
\hline
2.3 & Database Backup & The database backup is a copy of data from database that can be used to reconstruct that data. \\
\hline
2.4 & Transaction Log & It records all transactions and the database modifications made by each transaction. \\
\hline
\end{tabular}
\caption{Assets of SQLite}
\end{table}

\vspace{1cm}

\section*{Step 2: Create an architecture overview}
\setcounter{section}{2}
\subsection{Technologies}
\clearpage
\begin{table}[!htbp]
\centering
\begin{tabular}{>{\raggedright}m{2cm} m{10cm}}
\hline
Technology & Description \\
\hline
B-Tree & B-tree is a self-balancing tree data structure that keeps data sorted and allows searches, sequential access, insertions, and deletions in logarithmic time. \\
\hline
R-Tree & R-tree is a special index for the row of a query design. R-tree generally querier minimum or maximum rectangle in the xy coordinate GIS. \\
\hline
Tokenizer and Parser & The Tokenizer and Parser process the SQL statements and SQLite command, and convert them to SQLite internal byte codes that can be executed by Virtual Database Engine. \\
\hline
Page Cache & Because the available memory is limited, only a portion of the memory is used to store data from the database files. This part of the memory space is usually called a database cache or data buffer. In SQLite, it is called a page cache. \\
\hline
C Programming Language & C is a general-purpose, imperative computer programming language, supporting structured programming, lexical variable scope and recursion, while a static type system prevents many unintended operations. \\
\hline
\end{tabular}
\caption{Technologies used in SQLite}
\end{table}

\subsection{Architecture diagram}
\begin{figure}[!htbp]
\centering
\includegraphics[height=7.7cm]{arch.png}
\caption{Architecture of SQLite}
\end{figure}

\vspace{1cm}

\section*{Step 3: Decompose the application}
\setcounter{section}{3}
\setcounter{subsection}{0}
\subsection{Security Profiles}
\clearpage
\begin{table}[!htbp]
\centering
\begin{tabular}{>{\raggedright}m{3cm} m{13cm}}
\hline
Input Validation & \tabincell{m{13cm}}{1.	Not all the input is validated, users can type any SQL statements or other command to the front-end interface. Some invalid statements will be omitted by Tokenizer. Other statements, which may be malicious, will be executed. \\ 2.	Not all the data is validated as it is passed between separate trust bound. \\ 3. All the data in the database is created by user/program, so not all the data can be trusted.} \\
\hline
Authentication & SQLite runs only on local machine. Therefore, there are no strong account policies, no passwords enforced, no certificates, no password verifiers. \\
\hline
Authorization & SQLite does not have a user management system, which means every user can get access to the whole database file without authorization. \\
\hline
Configuration management & SQLite provides a set of APIs for users to configure database. These APIs can be called everywhere by any user. There are no specific security policies. \\
\hline
Sensitive data & \tabincell{m{13cm}}{1. All the SQL statements and database command are sensitive in the application. \\ 2. The user can get access to the whole database file. There are no specific secure policy to protect the file. And no encryption is enabled.} \\
\hline
Session management & Since no network are enabled, there is no session management. \\
\hline
Cryptography & The SQlite database is not responsible for encrypt the sensitive data and the data will be stored in plaintext. So there are no encryption algorithm in the SQLite database. \\
\hline
Parameter manipulation & \tabincell{m{13cm}}{1. The SQLite database itself does not detect tampered parameters since the applications that use SQLite will be responsible for detecting tampered parameters, such as a Web portal should implement the function to detect tampered parameters. \\ 2. The SQLite database itself does not validate all parameters in form fields, view state, cookie data, and HTTP headers.} \\
\hline
Exception management & \tabincell{m{13cm}}{1. When the developers implemented the SQLite, they considered most of possible exceptions in the C source code. \\ 2. Also, when the exception occurs, the SQLite database user can check the log files, such as the transaction logs to identify the exceptions. \\ 3. The exceptions are allowed to propagate back to the users since the log files are stored in the disk. \\ 4. The exception information will be recorded in the log files, which contain the exploitable information.} \\
\hline
Auditing and logging & \tabincell{m{13cm}}{1. The SQLite database stores log files(database files) in disk, which means that the attacks can be able to access these files by copying the disk of the system. \\ 2. Unlike most of database engines are client/server based, the SQLite is serverless. Any program that is able to access the disk is able to use the SQLite database.} \\
\hline
\end{tabular}
\caption{Security profiles in SQLite}
\end{table}



\setcounter{subsection}{1}
\subsection{Trust Boundaries, Data Flow and Entry Point}
\clearpage
\begin{figure}[!htbp]
\centering
\includegraphics[height=7.7cm]{arch_with_bo.png}
\caption{Trust Boundaries of SQLite}
\end{figure}

\vspace{1cm}

\section*{Step 4: Identify the threats}


\section*{Step 5: Document the threats}


\section*{Step 6: Rate the Threats}


\end{document}