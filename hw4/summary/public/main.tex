\documentclass[journal]{IEEEtran}


\hyphenation{op-tical net-works semi-conduc-tor}

\begin{document}

\title{Exploitation of Buffer Overflow Vulnerability}
\author{Qiyang~Gu, Ujjawal~Sharma, Zhiyuan~Li}
\maketitle


\begin{abstract}
Stack Smash Exploit is caused when a programme is trying to access (read or write) a region of an array that is not bounded by it. Suppose an array has length 'n' and the programme is trying to access its n+1\textsuperscript{th} location, in this case an overflow of stack will occur. This happens in the C and C++ programming languages. The reason behind such an occurance is that the runtime environment of these programming language do not provide a check on the buffer overflow. However, in other programming languages such as Java, Pascal, C-Sharp etc., the runtime environment do take care of buffer overflow. There are some functions in C such as strcpy and gets which does not include array bound checks and thus can lead to some major attacks. Attckers use this vulnerability to modify the execution of programme in the way they want. The attacker may take up the complete control of execution of the programme by using this vulnerability.
\end{abstract}

\begin{IEEEkeywords}
 stacksmash, bufferoverflow, vulnerability, exploit
\end{IEEEkeywords}


\section{Introduction}
\IEEEPARstart{F}or demonstrating the exploit, we have created our own set of C programmes. The whole package consists of three files. We have created C Code files namely vulcode.c, exploit.c and retlib.c. The C code file vulcode.c is the vulnerable code which is the original source code of the programme which the attacker will attack. The C code file exploit.c is the programme which will create a tainted file which will be helpful in creating the buffer overflow. The C code file retlib.c will act as the arc injection file. The attacker will try to pass the control of the execution from vulcode.c to retlib.c. 


\section{Reasons and Goals}
Performing the Stack Smash Vulnerability activity requires a mature knowledge of C programming language. By working on different exploit activities from some time we thought of creating our own programmes as we were confident on the end-to-end functionality of the working of exploit. Just by creating the above mentioned 3 files, one can demonstrate the stack smash vulnerability. Creating our own programme also helped us to understand what each step in a long programme can lead to. Also, by developing our code, we avoided the redundant codes of any huge programme. As we were working on the C exploits from a long time now, we thought of using all our previous experiences to create a new project altogether.

Below are the goals, which we had in our mind while performing this activity:
\begin{itemize}
\item To understand how the exploit works.
\item To understand how the control of the programme flows under such an activity.
\item To understand the way an attacker can use the vulnerabilities.
\item To get a good conceptual grip on what each step of exploit and vulnerable code means.
\item To understand how the control of any programme can be diverted.
\item To find out why some functions are vulnerable in C and the reason behind it.
\end{itemize}

\section{Project Execution Highlights}
\subsection{Implementing a stack smashing exploit that uses shellcode injection}
Firstly, we reviewed the knowledge provided to us in the class on how to exploit a bufferoverflow vulnerability and take advantage of this vulnerability to conduct stack smashing. We reviewed the structure of a stack and how it changes when it executes the function call procedure. Based on this, we were able to learn how to exploit stack smashing by code injection and arc injection and we aquired some basic understanding about how these two exploits work, and also, what's the difference between these two exploits. For the code injection, we constructed an exploit which will try to attack the vulnerable source code which has the unchecked \textit{strcpy()} or \textit{memcpy()} functions. The exploit will take the advantage of this unchecked function to pass argument of its choice.

Secondly, After we constructed a vulnerable source code that was containing unchecked \textit{strcpy()}, we wrote an exploit source code to conduct stack smashing. In this exploit suorce code, we constructed the machine code of shellcode and our goal was to rewrite the return address to our shellcode's address. We need to insert this malicious code into the stack and make sure the stack is executable. So, when we build the C source code, we set the compiler options as \textit{gcc -z execstack -fno-stack-protector} so that, the stack can be executable and the stack canaries would be closed.

Finally, we used the gdb to debug the program and got the information of ebp, esp, return address, and shellcode's address. Based on these information, we were able to do some calculation to determine what should we pass to the tainted buffer in order to make the vulnerable variable perform an overflow. This process is shown in detail in the Wiki. 

When the project successfully return to our shellcode rather than the original main function, we concluded that our exploit works. And then, we turned ON the ASLR and received the segment fault error while running the same program. This occured because the randomization of the address of heap, stack, code, and shared libraries can be achieved by ASLR. And when we turned ASLR OFF, and enabled stack canaries, the attack did not work since we changed the value of stack canaries when we overflow the vulnerable variable in the stack. This will lead to  triggering of the program exception handle mechanism. To defeat the stack protected method, such as DEP, we need to use arc injection, which we will describe in detail soon. That will explain how to attack the program without requiring an executable stack.

\subsection{Implementing a stack smashing exploit that uses arc-injection}
Firstly, we knew that this exploit is also a type of bufferoverflow attack. However, we don't require the stack to be executable in this case, and also we don't even need a shellcode. Instead, we let the vulnerable program jump to the existing code, such as the \textit{system()} function in the libc library that has been loaded into memory, to implement our attack. So, similar to the code injection, we constructed a vulnerable source code and build it, and construct an exploit source code and build it. In this case, we don't require to set the compiler option \textit{gcc -z exestack} as we would not insert any code into the stack.

Secondly, our goal is to let the vulnerable program jump to the \textit{system()} function and pass the \textit{/bin/sh} as argument. For this, we need to know the address of \textit{system()} and the address of environmental variable \textit{/bin/sh}. We can know the address of \textit{system()} by using \textit{p system} command in gdb since these functions are loaded into memory and their addresses are same for all the programs(As we've already turned off the ASLR). And, then, we will find the address of environmental variable \textit{/bin/sh} by \textit{getenv()}, or by using command \textit{x/500s \$esp} in gdb to find the address.

Finally, we tested the exploit program and it worked well. And, we let the vulnerable program jump to the \textit{system()} and open the /bin/sh successfully. This process is shown in detail in the Wiki. We also know that the effective way to prevent our program to face an arc injection attack is by simply turning ON the ASLR, since it will make the attck very difficult by randomizing the addresses of shared library. However, there are some more complicated method to achieve the attack, such as chaining return-to-libc calls via frame-faking method.

\section{Findings}
\subsection{Implementing a stack smashing exploit that uses shellcode injection}
The basic idea we got about code injection exploitation is that, when a function is called, we modify its return address in the stack to the address which is our shellcode address. Therefore, we need to get the exact address in the stack to put in our shellcode. If the ASLR (Address Space Layout Randomization) is off, we can calculate the address to be attacked by executing the program for several times. If the ASLR is on, the address in the memory will be randomized every time the program being executed. Therefore, we cannot figure out the address in the memory by executing the program, which means our injection will fail.

In our case, we turn on ASLR after we successfully implement the code injection. When we want to replay what we do previously, we find that we cannot figure out the address where we want to insert our shellcode because the address will change after we re-execute our program.

There are several ways to defeat ASLR, one of which is brute force. \cite{defeating_tech} We need to try lots of addresses until it works.

\subsection{Implementing a stack smashing exploit that uses arc-injection}
When the system enables DEP (Data Execution Prevent), the stack is actually ``non-executable''. \cite{dep} It means some specific space in memory will be marked as ``non-executable'' space. If we use basic code injection, the program will terminate when the system finds the shell code runs in the ``non-executable'' space. Therefore, what we do in the previous experiment will fail.

In order to defeat DEP, we use arc injection. The basic idea is that we add our malicious code to environment variables and system function instead of running it in the stack. In our attack, we transfer control flow of the program and change the stacker pointer to the system functions so that we can run our shell code.

To defeat arc injection, we should enforce CFI (Control Flow Integrity) to entrue the control flow of the program is not changed by the attacker.

\section{Conclusion}
In this homework, we understood the stack smash vulnerabilities and how an attacker can use it to divert the execution of any programme. Although to understand the whole process we took some time,but once we had understood the concept well, we were successfully able to execute it. We were able to understand the control flow of stack smashing vulnerability and how it is used by attackers to get the control of the programme. We were also able to understand why few of the functions are dangerous in C and C++ programming languages and why we should prefer not to use them.

\section{Appendix}
We put all supplementary information to the Wiki for our Gitlab project.

\begin{thebibliography}{10}
\bibitem{defeating_tech}
N. Stack Overflows - Defeating Canaries, ``Stack Overflows - Defeating Canaries, ASLR, DEP, NX'', Security.stackexchange.com. [Online]. Available: https://security.stackexchange.com/questions/20497/stack-overflows-defeating-canaries-aslr-dep-nx.

\bibitem{dep}
``Executable space protection'', En.wikipedia.org. [Online]. Available: https://en.wikipedia.org/wiki/Executable\_space\_protection.
\end{thebibliography}


\end{document}